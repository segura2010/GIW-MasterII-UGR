%%%%%%%%%%%%%%%%%%%%%%%%%%%%%%%%%%%%%%%%%
% Programming/Coding Assignment
% LaTeX Template
%
% This template has been downloaded from:
% http://www.latextemplates.com
%
% Original author:
% Ted Pavlic (http://www.tedpavlic.com)
%
% Note:
% The \lipsum[#] commands throughout this template generate dummy text
% to fill the template out. These commands should all be removed when 
% writing assignment content.
%
% This template uses a Perl script as an example snippet of code, most other
% languages are also usable. Configure them in the "CODE INCLUSION 
% CONFIGURATION" section.
%
%%%%%%%%%%%%%%%%%%%%%%%%%%%%%%%%%%%%%%%%%

%----------------------------------------------------------------------------------------
%	PACKAGES AND OTHER DOCUMENT CONFIGURATIONS
%----------------------------------------------------------------------------------------

\documentclass{article}

\usepackage{fancyhdr} % Required for custom headers
\usepackage{lastpage} % Required to determine the last page for the footer
\usepackage{extramarks} % Required for headers and footers
\usepackage[usenames,dvipsnames]{color} % Required for custom colors
\usepackage{graphicx} % Required to insert images
\usepackage{listings} % Required for insertion of code
\usepackage{courier} % Required for the courier font
\usepackage{lipsum} % Used for inserting dummy 'Lorem ipsum' text into the template
\usepackage{hyperref} % Hyperlink
\usepackage{enumitem}

\usepackage{float} % imagenes debajo de texto (http://tex.stackexchange.com/questions/85971/figure-goes-automatically-to-next-page)


\usepackage[utf8]{inputenc} % Para tildes!

% Margins
\topmargin=-0.45in
\evensidemargin=0in
\oddsidemargin=0in
\textwidth=6.5in
\textheight=9.0in
\headsep=0.25in

\linespread{1.1} % Line spacing

% Set up the header and footer
\pagestyle{fancy}
\lhead{\hmwkAuthorName} % Top left header
\rhead{\hmwkTitle} % Top center head
\lfoot{\lastxmark} % Bottom left footer
\cfoot{} % Bottom center footer
\rfoot{P�gina\ \thepage\ de\ \protect\pageref{LastPage}} % Bottom right footer
\renewcommand\headrulewidth{0.4pt} % Size of the header rule
\renewcommand\footrulewidth{0.4pt} % Size of the footer rule

\setlength\parindent{0pt} % Removes all indentation from paragraphs


%----------------------------------------------------------------------------------------
%	DOCUMENT STRUCTURE COMMANDS
%	Skip this unless you know what you're doing
%----------------------------------------------------------------------------------------


%----------------------------------------------------------------------------------------
%	NAME AND CLASS SECTION
%----------------------------------------------------------------------------------------

\newcommand{\hmwkTitle}{Gesti�n de Informaci�n en la Web} % Assignment title
\newcommand{\hmwkSubtitulo}{Master en Ingenier�a Inform�tica} % Due date
\newcommand{\hmwkDueDate}{Martes 5 de Abril de 2016} % Due date
\newcommand{\hmwkTituloTarea}{Pr�ctica 2: An�lisis y Evaluaci�n de Redes en Twitter} % Class/lecture time
\newcommand{\hmwkAuthorName}{Luis Alberto Segura Delgado} % Your name
\newcommand{\hmwkAuthorEmail}{segura2010@correo.ugr.es} % Your email
\newcommand{\hmwkAuthorDNI}{45922174-Y} % Your email

\renewcommand{\figurename}{Figura}
\renewcommand{\contentsname}{�ndice}

%----------------------------------------------------------------------------------------
%	TITLE PAGE
%----------------------------------------------------------------------------------------

\title{
\vspace{2in}
\textmd{\textbf{\hmwkTitle}}\\
\textmd{\textbf{\hmwkSubtitulo}}\\
\normalsize\textbf{\\\hmwkTituloTarea}\\
\vspace{0.1in}
\vspace{3in}
}

\author{\textbf{\hmwkAuthorName} \\
	\textbf{DNI}: \hmwkAuthorDNI \\
	\texttt{\hmwkAuthorEmail}
}
\date{\hmwkDueDate} % Insert date here if you want it to appear below your name

%----------------------------------------------------------------------------------------

\begin{document}
\inputencoding{latin1}
\maketitle


%----------------------------------------------------------------------------------------
%	PROBLEM 1
%----------------------------------------------------------------------------------------

% To have just one problem per page, simply put a \clearpage after each problem

\newpage

\tableofcontents

\newpage

\section{Introducci�n}

El objetivo de esta segunda pr�ctica es formalizar todos los conocimientos adquiridos en el curso aplic�ndolos a un caso real de an�lisis de una red social online generada a partir de un medio social. Para ello, se ha seleccionado un medio social concreto (Twitter) y una pregunta de investigaci�n. A partir del medio social elegido, se obtendr� el conjunto de datos y se construir� una red social, que ser� analizada con objetivo de responder a la pregunta de investigaci�n planteada.


\section{Trabajo Realizado}

En esta secci�n se detalla el trabajo realizado en la pr�ctica, indicando en primer lugar el problema concreto que se ha planteado y el conjunto de datos y la forma de obtenerlos para resolver dicho problema. A continuaci�n se explicar� el an�lisis realizado sobre los datos y la red social obtenida y finalmente las conclusiones obtenidas del estudio.

\subsection{Descripci�n del Problema}

El problema a estudiar es detectar cuales son los usuarios m�s relevantes en la discusi�n de Twitter sobre la emisi�n en \textbf{Periscope}\footnote{\url{https://www.periscope.tv}} que tuvo lugar el d�a 25 de marzo, organizada por Gerard Piqu�\footnote{\url{http://as.com/videos/2016/03/25/portada/1458916408_738738.html}}.
\\
Para abordar el problema, se han recopilado tweets publicados durante la emisi�n en los que se mencionaba a Piqu� (@3gerardpique) y se inclu�a la palabra "Periscope". Y como la obtenci�n de los datos se realiz� unos d�as despu�s, se han limitado la b�squeda a los tweets que se publicaron el d�a 25 de Marzo, d�a de la emisi�n\footnote{B�squeda avanzada de Twitter: @3gerardpique periscope since:2016-03-25 until:2016-03-26 (\url{https://twitter.com/search?vertical=default&q=\%403gerardpique\%20periscope\%20since\%3A2016-03-25\%20until\%3A2016-03-26&src=typd})}. Para obtener los tweets, se ha utilizado la herramienta NodeXL.
\\\\
De cara a evaluar la red correctamente, se ha decidido eliminar el nodo de Piqu� de la red, pues todos los tweets lo mencionan, por tanto se conecta con todos los usuarios, y esto dificulta el an�lisis de la red y su visualizaci�n al mismo tiempo que no resulta interesante.

\subsection{C�lculo de los valores de las medidas de an�lisis}

\subsection{Propiedades de la red}

\subsection{Calculo de los valores de las medidas de an�lisis de redes sociales}

\subsection{Descubrimiento de comunidades}

\subsection{Visualizaci�n de la red social}

\subsection{Discusi�n de los resultados y Conclusiones}

\end{document}